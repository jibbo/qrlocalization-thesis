\chapter{Conclusions}

This thesis describes a method to estimate the position of a QRcode inside a configured environment.
The dissertation stated the need of this particular application at the beginning and continued with an analysis of the tools used. Afterwards, the thesis explained the solving algorithm leading to the demonstration of its capability with tests. Therefore, these results look promising and can be used as a base to build an indoor step-by-step navigator. In particular, this was done with the goal to free elders of their frighten of new and crowded place as stated by the DALi project. Furthermore, this system is actually being used on a testing rollator, in order to discover its ability to let a user explore an unfamiliar environment. All the code is public on my GitHub's \footnote{ \url{http://github.com/jibbo/qrlocalization-thesis} } account and it is released under the license Creative Commons CC BY-NC 4.0 \footnote{\url{http://creativecommons.org/licenses/by-nc/4.0/}} and any form of redistribution is welcomed as far as DALi is mentioned and credited.\newline
Future work might include:
\begin{itemize}
	\item Find and solve the problem with ZBar recognition which appears when a QRCode has certain orientations. This would improve the whole system capabilities.
	\item Improvements on the rectification algorithm are needed in order to remove duplicate code.
	\item Make experiments with QRCodes printed with \textbf{invisible ink} and an infrared camera, in order to free the system by light problems and paper's decay giving by usage. 
\end{itemize}       