\chapter{An overview}

\vspace{5cm}
In modern times the world moves forward with incredible speed and creates new buildings aimed to engage people who use them, but often these environments are designed for adults with fully developed psychological and physical capabilities.
People who don't have the necessary requirements to navigate properly these new places begin to feel unsafe and perceive the world around even with hostility, which led to a confinement in buildings they know and trust.
This confinement, which became like a prison, reduces a variety of chances, such as: physical exercise, the gathering of fresh food and the making of social activities.

\newpage

\section{The DALi project}

\vspace{1cm}
\begin{center}
      \includegraphics[width=0.3\textwidth]{img/Dali-logo.png}
\end{center}
\vspace{1cm}

A specific subset of people who experience that kind of problems every day are elders and, in order to provide help for this part of population, DALi is researching new technology which can effectively impact on their quality of living by simulating and validating with experiments on the field.
\newline
The major focus of the project at the moment of writing is a robotic wheeled walker which can help both the psychological sphere, by providing freedom of movement in newly discovered places to the user, and the physical one by actually aiding him during actual movements.

\section{Role of the thesis inside DALi}
\vspace{0.5cm}
Designing and building a rollator brings many challenges in the field of robotics, and even in the human computer interaction one, which are very interesting but, at the same time, they may reveal very difficult to develop. 
In fact, at the core of the walker functionalities there is a localization system which can track the user's movements in order to guide the person to his destination.
The precision requirement of the above mentioned unit is an extremely delicate matter and needed further investigation in order to bring the best possibilities whilst limiting the error.
Therefore the work described on the next pages explores the possibility of building an environment where qr-codes provide information about the position and the trajectory to the rollator which can translate them in actual guidance for the user.






  
