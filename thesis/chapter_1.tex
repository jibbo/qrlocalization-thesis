\chapter{Introduction}
In modern times the world moves forward with incredible speed and creates new buildings aimed to engage people who use them, but often these environments are designed for adults with fully developed psychological and physical capabilities.
People who don't have the necessary requirements to navigate properly these new places begin to feel unsafe and perceive the world around even with hostility, which led to a confinement in buildings they know and trust.
This confinement, which become like a prison, reduces a variety of chances, such as: physical exercise, the gathering of fresh food and the making of social activities.
A specific subset of people who experience this kind of problems every day are elders and, in order to freeing them of this burden, an indoor localization system for a wheeled walker is being researched . 

\section{Motivations}

There are many technical reasons to work in a project for assisted living, every one of them is a challenge and you can be considered a pioneer if your job is well-done.
However, personal satisfaction wasn't my primary goal when I had chosen this thesis and, whereas my colleagues felt the breeze derived from exploring a relatively new field, what I wanted to find was a solution to help real people who needs a "hand" from technology.
I realized that now it is the time to apply my effort in a product that is meaningful for a category of people who suffer. 
In fact, I could only imagine how hard is, for an old person who enjoyed his liberty when young, to be limited and have to rely on other people.
I can not say that I succeed because my work is just a small brick on the wall that this magnificent subject put in front of us, but I hope that my work would be a foundation for further work and that in a future not very far, a tool to assist elders will be invented .
For this, connected to the will of improving my skills, I undertook this path.

\section{The DALi project}

\vspace{1cm}
\begin{center}
      \includegraphics[width=0.3\textwidth]{img/Dali-logo.png}
\end{center}
\vspace{1cm}

The name stands for Devices for Assisted Living, which means the all the efforts of the team are focused in producing tools which helps people with deambulation or sensory problem in crowded places.
The project started in 2013 and it is composed by a consortium of universities and national research centers based in the following countries of Europe: United Kingdom, Spain, Greece, France and Austria.
Every partner contributes in specific areas of the project which, for example, involves both user centric design and human machine interfaces but also sensitive or cognitive algorithms.
The whole work is coordinated by The University of Trento and in particular by professor Luigi Palopoli.
At the moment of this writing, DALi is researching a new robotic wheeled walker, called rollator, which can help both the psychological sphere, by providing freedom of movement in newly discovered places to the user, and the physical one by actually aiding him during real movements.


\section{Role of the thesis inside DALi}
At the core of the walker functionalities there is a localization system which can track the user's position in order to provide directions to the user once processed.
The precision requirement of the above mentioned unit is an extremely delicate matter and needed further investigation.
Therefore the work described on the next pages explores the possibility of building an environment where qr-codes are the main providers of information about location and trajectory of the user.

\section{Goals of the thesis}
During the development of the project there were many sub-goals in order to achieve a final result although they were too numberouse to report it.
Therefore the followings are the main goals which the thesis achieved.
\begin{itemize}
	\item Show that the realization of this systems is possible
	\item Show that the usage of this systems is possible
	\item Show how to find a QRCode's angle relative to the camera
	\item Show how to improve QRCode recognizement of the ZBar library.
\end{itemize}


\section{Outline}
The second chapter of the thesis will describe the problem and explain why a marker has been chosen.
The third chapter contains all the information about the tools used by this project while
the fourth chapter will deeply analyze the algorithm.
In addition, chapter 5 and 6 show experimental proofs of accuracy of the solution on its own system and also on the rollator.
The last chapter states the conclusions about the work of the thesis.






  
