\chapter{An overview}

\vspace{0.5cm}
\subsubsection{The DALi project}

\vspace{1cm}
\begin{center}
      \includegraphics[width=0.3\textwidth]{img/Dali-logo.png}
\end{center}
\vspace{1cm}

In modern times the world moves forward with incredible speed and creates new buildings aimed to engage people who use them, but often this environments are designed for adults with fully developed psychological and physical capabilities.
People who don't have the requirements to navigate properly in this new places begin to feel unsafe and perceive the world around even with hostility, which brings them confined in buildings they know and trust.
The confinement reduces a variety of chances, such as: physical exercise,
fresh food supply and social activities.
\newline
A specific subset of people who experience this kinds of problems every day are elders and, in order to provide help for this part of population DALi is researching new technology which can effectively impact on their quality of living.
\newline
The major focus of the project at the moment of writing is a rollator which can help both the psychological sphere, by providing freedom of movement to the user, and the physical one by actually aiding him during actual movement.

\newpage
\subsubsection{Role of the thesis inside DALi}
\vspace{0.5cm}
Designing and building a rollator brings many challenges in the robotic field, and even in the human computer interaction one, which are very interesting but, at the same time, they may reveal very difficult to develop. 
In fact, at the core of the rollator's functionalities there is a localization system which can track the user's movements in order to guide the person to his destination.
The precision of the above mentioned unit is an extremely delicate matter and that's why it needed further investigation in order to bring the best possibilities while limiting the error.
Therefore the work described on the next pages explores the possibility of building an environment where qr-codes provides information about the position and the trajectory to the rollator which can translate them in actual guidance for the user. 






  
