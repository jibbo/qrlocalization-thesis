\chapter{General problem description}

In order to build the new rollator with the capability of giving directions to
the user and make him less ascared of unfamiliar places there are the following
two problems to solve.

\section{Indoor positioning system}
The first problem encountered is the localization of the elder inside a
building, and there are many solution available online which offer indoor 
GPS systems to try and buy.
However this platforms are based on a triangulation of Wi-fi signals united a
complete map of available hostspots.
\footnote{Further info at: 
\url{https://en.wikipedia.org/wiki/Wi-Fi_positioning_system}}
These kind of systems are full-stack solution to the problem and don't have the
necessary flexibility for the project's needs and, in addition, wi-fi hostspots 
aren't a possibility because the installation process would be a heavy burden
for user.
\newline
The team developing the project thought to use some sort of computer readable 
marker positioned inside the environment which contained the necessary information
and from this decision derived the usage of QR Codes, which I studied and
developed.

\newpage
\subsection{QR Code generalities}
A QRCode is foundamentally an image containing an encoded binary matrix of data.
According to the quantity of information which needs to be stored 
%The matrix module is called "version" and it adds 4 bits at each step from a
%minumimum of 21x21 to a maximum of 177x177 according to the information 
%length which has to be stored, until a maximimum of 23648 bits.

\footnote{with an L level or error correction}
\footnote{complete table at: \url{http://www.qrcode.com/en/about/version.html}}    














  
